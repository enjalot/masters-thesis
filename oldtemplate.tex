\documentclass[letter,11pt]{article}

\usepackage{ulem}
\usepackage{verbatim}
%\linespread{1.6}  % double spaces lines
\usepackage[left=1in,top=1in,right=1in,bottom=1in,nohead]{geometry}

% Temporarily put all figures and tables at the end of the document
%\usepackage[nolists,]{endfloat}
%\renewcommand{\efloatseparator}{\mbox{}}

\usepackage{amsmath, amssymb, graphics}
\usepackage[pdftex]{graphicx}
\usepackage[colorlinks, bookmarksopen, backref,
            pdfauthor={Ian D. Johnson},
            pdftitle={Real-Time Particle Simulations for Educational Games with OpenCL and Blender},
            pdfcreator={pdftex},
            pdfsubject={algorithms},
            linkcolor={blue},
            anchorcolor={black},
            citecolor={green},
            filecolor={magenta},
            menucolor={blue},
            pagecolor={red},
            plainpages=false,pdfpagelabels,
            urlcolor={cyan}]{hyperref}
\usepackage{hypcap}

%\usepackage{mathpazo}
\usepackage{flexisym}
\usepackage{breqn}
\usepackage[numbers,sort&compress]{natbib}
\usepackage{hypernat}

\newcommand{\mathsym}[1]{{}}
\newcommand{\unicode}[1]{{}}

\input{macros.tex}
\input{misc_mac.tex}

% Speaker's name, Affiliation, Email address, Title of talk, Co-authors, Abstract (preferably in text/plain TeX form, up to 150 words)

\title{\bf Real-Time Particle Simulations for Educational Games with OpenCL and Blender}

\author{Ian D. Johnson \\ idj03@fsu.edu \\ Florida State University \and Gordon Erlebacher \\ gerlebacher@fsu.edu \\ Florida State University }
%Florida State University, Dept. of Scientific Computation, Dirac Science Library Tallahassee, FL 32304}
\begin{document}

%%%  PROVIDED BY PACKAGE ulem %%%%%
% http://www.josterpi.com/latex-notes.html
% \uline{} - underline
% \uuline{} - double underline
% \uwave{} - wavy underline
% \sout{} - strikeout
% \xout{} - scratchout(?) (lots of diagonal lines)
%%%%%%%%%%%%%%%%%%%%%%%%%

\maketitle
\tableofcontents
\listoffigures
\listoftables 

\begin{abstract}
    asdf
\end{abstract}

\section{Introduction}


\section{Educational Game Development}
\subsection{Blender}


\section{GPU Computing}
\subsection{OpenCL}


\section{Particle Systems}


\section{Fluid Simulation}
\subsection{SPH}

\section{Implementation}


\section{Results}


\section{Future Work}



%\pagebreak
%%%%REFERENCES
%\bibliographystyle{acm}
%\bibliographystyle{plainnat}
%\bibliography{/panfs/panasas1/users/idj03/research/iansvn/cnpaper2010/references}

%%%%\bibliographystyle{unsrtnat}
%%%%\bibliography{thesis}

\section{Appendix}
\subsection{SPH Parameters}
\subsection{Source Code}


\end{document}
