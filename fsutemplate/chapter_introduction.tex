\chapter{Introduction}
\label{chapter:introduction}

Advances in computational power have lead to many developments in science and
entertainment. Powerful simulations which required expensive supercomputers can
now be carried out on a consumer personal computer and many children and young
adults spend countless hours playing sophisticated computer games. The focus of
this research is the development of tools which can help bring the entertaining
and appealing traits of video games to scientific education.


Video games have seen remarkable growth in recent times, and are one of the
largest entertainment industries \cite{Kafai2006}. Interactive games are played by
children and adults of all ages, on specialized consoles, on workstations and
even telephones. The pervasiveness of games in our culture is not in dispute,
their effect however, is.\cite{Ferguson2007} Parents raise concern over the amount of time
children spend playing games, and what their children may be learning from the
content in these games. The aim of this research is to help develop games which
encourage interest in science and teach scientific concepts, methodology and
thought processes. Applying video game technology to education is an ongoing
area of research and practice\cite{Hayes2008}. 


Tools for building games are varied and many, ranging from complex 3D modeling
programs, sophisticated physics simulators and design authoring programs. Many
professional tools costing thousands of dollars are used by the gaming industry
today, while there exist competitive free and open source projects supported by
everyone from hobbyists to major studios. These free and open source projects
are quickly catching up to their professional counterparts and often have an
extensive community surrounding their use and development. Educators wishing to
develop games are not likely to have the large financial resources required to
produce a successful game, but using the free tools available fun and
captivating games can be constructed. These tools include the Blender 3D
content creation suite and game engine, the Bullet physics engine, audio
authoring programs such as Audacity and the 2D media editor GIMP. 


One tool missing from this free toolbox is an interactive fluid simulator.
Fluid simulations have been increasingly used by the film and entertainment
industry for better special effects in movies and television, and the game
industry has been following suit. Fluid simulation is computationally expensive
and difficult to achieve at the speeds needed in game development. 

Recent developments in graphics programming and hardware have made it feasible
to simulate convincing fluids at frame rates suitable for games. The GPU was
developed as a way to accelerate the 3D graphics rendering used in PC video
games and has been a huge commercial success. A modern high-end consumer GPU
has the same computational power as a supercomputer from 10 years prior.
Methods from Computational Fluid Dynamics developed for supercomputers could be
adapted to work on a consumer PC.


The goal of this research is to develop an open source tool implementing the
Smoothed Particle Hydrodynamics (SPH) method for fluid simulation that can be
used in developing educational games. First an overview of game development
with the Blender software suite is given in Chapter 2.  
Chapter 3 introduces the concepts behind GPU computing. Particle systems in
general are covered in Chapter 4 followed by the theory behind the SPH method
in Chapter 5. The implementation details of the project are given in Chapter 6,
and the results of the effort described in Chapter 7. The thesis concludes with
an overview of future work planned for the project in Chapter 8. 


\section{Related Work}

Interactive fluid simulation for use in games draws heavily from Computational
Fluid Dynamics (CFD) and Computer Graphics research. The use of CFD methods for
film and animation spans two decades\cite{Tan2009} and has improved in
performance tremendously. One of the most common techniques for interactive
simulations of fluids is the Smoothed Particle Hydrodynamics (SPH) method
because of its relatively low cost and high flexibility. It was adapted for use
in computer animation by Desbrun et al. in 1996.\cite{Desbrun1996}. 


Blender, a popular 3D content creation suite, and the software used as a
platform for this thesis includes two fluid simulators, one based on SPH\cite{Clavet2005} and
one based on the Lattice Boltzmann Method\cite{Nils2006}. Neither of these simulators are
designed to run in real-time, nor are they available in the Blender Game
Engine. 


In 2003 Muller et al. used SPH to achieve 5000 particles at 5 frames per
second\cite{Muller2003}.  Others implemented interactive SPH simulations with
various improvements and additions to the algorithm with similar
performance.\cite{Kallin2009}\cite{Kelager2006}
With the advent of GPU computing, researchers capitalized on the parallel
implementations of SPH and were able to achieve real-time results with 16,000 particles.\cite{Harada2007a}
Many types of particle systems have been accelerated with the GPU, and several issues
important to general particle systems such as accessing values of neighboring
particles are important in SPH.\cite{Knowles2009}
The CUDA programming language made general purpose programming on the GPU more
accessible and researchers were able to achieve upwards of 60 frames per second
for simulations involving 60,000 interacting particles.\cite{Seizinger2010}
Recently SPH was implemented in CUDA with complex simulations able to handle
128,000 particles in real-time and simple simulations with upwards of 500,000
particles.\cite{Krog2010}. Real-time simulation is only part of the goal of making
realistic interactive fluid animations. Techniques for rendering the
surface of a fluid have been studied along with improvements to SPH in CUDA 
produced convincing animations with 75,000 particles.\cite{Goswami2010} 


For further historical reviews of SPH implementations see Hoetzlein et.
al\cite{Hoetzlein} and Krog's Master's Thesis.\cite{Krog2010}



