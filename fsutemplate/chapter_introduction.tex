\chapter{Introduction}

Advances in computational power have lead to many developments in science and
entertainment. Powerful simulations which required expensive supercomputers can
now be carried out on a consumer personal computer and many children and young
adults spend countless hours playing sophisticated computer games. The focus of
this research is the development of tools which can help bring the entertaining
and appealing traits of video games to scientific education.


Video games have seen remarkable growth in recent times, and are one of the
largest entertainment industries \cite{needed}. Interactive games are played by
children and adults of all ages, on specialized consoles, on workstations and
even telephones. The pervasiveness of games in our culture is not in dispute,
their effect however, is. Parents raise concern over the amount of time
children spend playing games, and what their children may be learning from the
content in these games. It is our desire then to help develope games which
encourage interest in science and teach scientific concepts, methodology and
thought processes. Applying video game technology to education is an ongoing
are of research and practice\cite{needed}. 


Tools for building games are varied and many. For the purpose of education we
strive for the widest possible adoption, hence we emphasize free and open
tools. 


The proliferation of high-end consumer graphics cards for use in gaming has had
an unintended consequence for science. Graphics cards are designed to perform
specific types of operations quickly and in parallel. These operations are also
used in many scientific applications.  .\cite{OpenCL}


\section{Related Work}

Krog - CUDA SPH
Blender - SPH + LBM (not real-time)

Blender, a popular 3D content creation suite, and the software used as a
platform for this thesis includes two fluid simulators, one based on SPH\cite{blenderSPH} and
one based on the Lattice Boltzmann Method\cite{Nils}. Neither of these simulators are
designed to run in real-time, nor are they available in the Blender Game
Engine. 


Harada et al
Other SPH papers
