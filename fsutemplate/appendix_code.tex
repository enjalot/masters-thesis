
\section{Download}
\label{appendix:download}
The code for the entire modified Blender project is available for download at the following url \\
\url{https://github.com/enjalot/BGERTPS} \\
If git\cite{Git} is installed, the following command will
download the code: 
\begin{verbatim}
git clone git://github.com/enjalot/BGERTPS.git
cd BGERTPS
git submodule init
git submodule update
\end{verbatim} 

The code for just the RTPS library is available for download at the following url \\ 
\url{https://github.com/enjalot/EnjaParticles} \\
If git\cite{Git} is installed, the following command will
download the code: 
\begin{verbatim}
git clone git://github.com/enjalot/EnjaParticles.git
\end{verbatim} 

Build instructions for each platform are provided in the README and INSTALL
files in the root directory of both source trees.


\section{SPHParams}
\label{appendix:sphparams}
The structure used to pass physical and simulation parameters to OpenCL.
\begin{cppcode}{0}

#ifdef WIN32
#pragma pack(push,16)
#endif
typedef struct SPHParams
    {
        float mass;
        float rest_distance;
        float smoothing_distance;
        float simulation_scale;

        //dynamic params
        float boundary_stiffness;
        float boundary_dampening;
        float boundary_distance;
        float K;        //gas constant

        float viscosity;
        float velocity_limit;
        float xsph_factor;
        float gravity; // -9.8 m/sec^2

        float friction_coef;
        //next 4 not used at the moment
        float restitution_coef;
        float shear;
        float attraction;

        float spring;
        //float surface_threshold;
        //constants
        float EPSILON;
        float PI;       //delicious
        //Kernel Coefficients
        float wpoly6_coef;

        float wpoly6_d_coef;
        float wpoly6_dd_coef; // laplacian
        float wspiky_coef;
        float wspiky_d_coef;

        float wspiky_dd_coef;
        float wvisc_coef;
        float wvisc_d_coef;
        float wvisc_dd_coef;


        //CL parameters
        int num;
        int nb_vars; // for combined variables (vars_sorted, etc.)
        int choice; // which kind of calculation to invoke
        int max_num;

 } SPHParams
#ifndef WIN32
    __attribute__((aligned(16)));
#else
        ;
        #pragma pack(pop)
#endif

\end{cppcode}

\section{Cell Indices Kernel}
\label{appendix:cellindices}
The OpenCL Kernel for computing the start and end indices pointing into the
particle array for each grid cell.

\begin{cppcode}{0}
#include "cl_macros.h"
#include "cl_structs.h"
        
__kernel void cellindices(
                            int num,
                            __global uint* sort_hashes,
                            __global uint* sort_indices,
                            __global uint* cell_indices_start,
                            __global uint* cell_indices_end,
                            //__constant struct SPHParams* sphp,
                            __constant struct GridParams* gp,
                            __local  uint* sharedHash   // blockSize+1 elements
                            )
{
    uint index = get_global_id(0);
    uint ncells = gp->nb_cells;

    uint hash = sort_hashes[index];
    //don't want to write to cell_indices arrays if hash is out of bounds
    if( hash > ncells)
    {   
        return;
    }   
    // Load hash data into shared memory so that we can look 
    // at neighboring particle's hash value without loading
    // two hash values per thread   
    uint tid = get_local_id(0);
    sharedHash[tid+1] = hash;  // SOMETHING WRONG WITH hash on Fermi

    if (index > 0 && tid == 0)
    {
        // first thread in block must load neighbor particle hash
        uint hashm1 = sort_hashes[index-1] < ncells ? sort_hashes[index-1] : ncells;
        sharedHash[0] = hashm1;
    }

#ifndef __DEVICE_EMULATION__
    barrier(CLK_LOCAL_MEM_FENCE);
#endif

    // If this particle has a different cell index to the previous
    // particle then it must be the first particle in the cell,
    // so store the index of this particle in the cell.
    // As it isn't the first particle, it must also be the cell end of
    // the previous particle's cell

    //Having this check here is important! Can't quit before local threads are done
    //but we can't keep going if our index goes out of bounds of the number of particles
    if (index >= num) return;

    if (index == 0)
    {
        cell_indices_start[hash] = index;
    }

    if (index > 0)
    {
        if(sharedHash[tid] != hash)
        {
            cell_indices_start[hash] = index;
            cell_indices_end[sharedHash[tid]] = index;
        }
    }
    //return;

    if (index == num - 1)
    {
        cell_indices_end[hash] = index + 1;
    }
}

\end{cppcode}

\section{Density Kernel}
\label{appendix:density}
The OpenCL Kernel for computing the density of each particle.

\begin{cppcode}{0}
//These are passed along through cl_neighbors.h
//only used inside ForNeighbor defined in this file
#define ARGS __global float4* pos, __global float* density
#define ARGV pos, density
        
#include "cl_macros.h"
#include "cl_structs.h"
//Contains all of the Smoothing Kernels for SPH
#include "cl_kernels.h"
        
inline void ForNeighbor(//__global float4*  vars_sorted,
                        ARGS,
                        PointData* pt,
                        uint index_i,
                        uint index_j,
                        float4 position_i,
                        __constant struct GridParams* gp,
                        __constant struct SPHParams* sphp
                        DEBUG_ARGS
                       )
{
    int num = sphp->num;
    float4 position_j = pos[index_j] * sphp->simulation_scale; 
    float4 r = (position_i - position_j);
    r.w = 0.f; // I stored density in 4th component
    // |r|
    float rlen = length(r);

    // is this particle within cutoff?
    if (rlen <= sphp->smoothing_distance)
    {
        // return density.x for single neighbor
        float Wij = Wpoly6(r, sphp->smoothing_distance, sphp);

        pt->density.x += sphp->mass*Wij;
    }
}
//Contains Iterate...Cells methods and ZeroPoint
#include "cl_neighbors.h"

//--------------------------------------------------------------
// compute forces on particles

__kernel void density_update(
//                       __global float4* vars_sorted,
                       ARGS,
                       __global int*    cell_indexes_start,
                       __global int*    cell_indexes_end,
                       __constant struct GridParams* gp,
                       __constant struct SPHParams* sphp
                       DEBUG_ARGS
                       )
{
    // particle index
    int nb_vars = sphp->nb_vars;
    int num = sphp->num;
    //int numParticles = get_global_size(0);
    //int num = get_global_size(0);

    int index = get_global_id(0);
    if (index >= num) return;

    float4 position_i = pos[index] * sphp->simulation_scale;

    //debuging
    clf[index] = (float4)(99,0,0,0);
    //cli[index].w = 0;

    // Do calculations on particles in neighboring cells
    PointData pt;
    zeroPoint(&pt);

    //IterateParticlesInNearbyCells(vars_sorted, &pt, num, index, position_i, cell_indexes_start, cell_indexes_end, gp,/* fp,*/ sphp DEBUG_ARGV);
    IterateParticlesInNearbyCells(ARGV, &pt, num, index, position_i, cell_indexes_start, cell_indexes_end, gp,/* fp,*/ sphp DEBUG_ARGV);
    density[index] = sphp->wpoly6_coef * pt.density.x;
    /*
    clf[index].x = pt.density.x * sphp->wpoly6_coef;
    clf[index].y = pt.density.y;
    clf[index].z = sphp->smoothing_distance;
    clf[index].w = sphp->mass;
    */
    clf[index].w = density[index];
}

\end{cppcode}

