\documentclass[11pt]{fsuthesis}
% Imports provided with Evan's template
%\usepackage[overload]{textcase}
%\usepackage[dvips]{graphics}
\usepackage{hyperref}
%\hypersetup{breaklinks=true}

% Imports added by Ian
\usepackage{verbatim}
\usepackage{amsmath, amssymb, graphics}
\usepackage[pdftex]{graphicx}
\usepackage[numbers,sort&compress]{natbib}
\usepackage{textcomp} %supposedly needed by listings
\usepackage{subfig}






\input{macros.tex}
\input{misc_mac.tex}


\title{Real-Time Particle Systems in the Blender Game Engine}
\author{Ian Johnson}
\college{College of Arts and Sciences}
\department{Department of Scientific Computing}
\manuscripttype{Thesis}
\degree{Master of Science}
\semester{Summer}
\degreeyear{2011}
\defensedate{May 1, 2010}

\committeesig{Gordon Erlebacher}{Professor Directing Thesis}
\committeesig{A}{University Representative}
\committeesig{Tomasz Plewa}{Committee Member}
\committeesig{Anter El-Azab}{Committee Member}

\collegesig{Max Gunzburger}{Chair}{Department of Scientific Computing}
\collegesig{Ima Bigwig}{Dean}{College of Arts and Sciences}

\begin{document}

\frontmatter
\maketitle
\makesignaturepage

%\begin{dedication}
%\end{dedication}

%\begin{acknowledgments}
%\end{acknowledgments}

\tableofcontents
%\listoftables
\listoffigures
%\listofmusex

%\begin{listofsymbols}
%\end{listofsymbols}

%\begin{listofabbrevs}
%\end{listofabbrevs}

\begin{abstract}
\begin{figure}[!htc]
 		\centering
		\includegraphics[scale=0.15]{figures/dsc_fsuseal.png}
		\label{fig:dsc_fsuseal}
\end{figure}

Advances in computational power have lead to many developments in science and
entertainment. Powerful simulations which required expensive supercomputers can
now be carried out on a consumer personal computer and many children and young
adults spend countless hours playing sophisticated computer games. The focus of
this research is the development of tools which can help bring the entertaining
and appealing traits of video games to scientific education.

One of the main factors contributing to the rise in computational power and the
increasing sophistication of video games is the Graphics Processing Unit (GPU).
As a piece of consumer hardware a GPU is relatively low cost and available to
almost anyone with a computer. The market for GPUs has been primarily driven by
video gamers wanting to improve the experience of their games through better
and often more realistic graphics. 

Video game developers use many tools and programming languages to build their
games, among these tools is the Blender 3D content creation suite. Blender
includes a Game Engine which can be used to design and develop sophisticated
interactive experiences. One important tool in computer graphics and animation
is the particle system, particle systems can be used to simulate effects such
as fire and smoke as well as provide realistic simulations of fluids like water
or oil. The particle system available in Blender is unfortunately not available
in the Blender Game Engine because it is not fast enough to run in real-time.

This thesis presents a particle system library accelerated by the GPU using the
OpenCL programming language. The primary system implemented in this research is
a fluid simulator using the Smoothed Particle Hydrodynamics technique for
simulating incompressible fluids. The library is integrated into the Blender
Game Engine providing an interactive platform for exploring fluid dynamics and
creating video games with realistic water effects.


\end{abstract}

\mainmatter

\input chapter_introduction
\input chapter_games
\input chapter_gpu
\input chapter_particles
\input chapter_sph
\input chapter_implementation
\input chapter_results
\input chapter_future

\pagebreak

\appendix
\input appendix_code
\input appendix_modified

% You have your choice of bibliography sections, either
% hand-crafted or BibTeX.

% This is the "hand-crafted" bibliography/references section:
%\begin{references}
%Mybib, Sample.  \textit{An Example of a Bibligraphic Entry
% Created Manually}.  Tallahassee, Florida: Fornish and Frak, 2010.
%
%Smith, Marigold.  \textit{Lots and Lots of Bibliographic Entries
% and How to Display Them}.  Tallahassee, Florida: Gibson and Goulash, 2010.
%\end{references}

% The BibTeX bibliography/references section.  View the file
% 'myrefs.bib' to get a feel for what these entries may look like.
%\bibliographystyle{plain}
\bibliographystyle{unsrtnat}
\bibliography{mythesis}

\begin{biosketch}
    Ian 'enjalot' Johnson was born on the 6th of October, 1985 in Leiden,
    Netherlands. Ian grew up in Tallahassee, Florida where he became interested
    in computers at a young age. One chapter of his youth involved lucrative
    yet misguided efforts in hacking video games. Upon reaching some semblance
    of maturity Ian has channeled his creative energies into constructive
    projects and now enjoys building tools that help people create video games.

    Ian's research interests include computer graphics, scientific
    visualization, and interactive education. When not studying, he enjoys
    spending time with his loved ones, teaching those around him and traveling
    and eating local foods.

    Ian currently lives in Tallahassee, Florida but can be found from anywhere
    in the world by visiting his web-site \url{http://enja.org}.
\end{biosketch}

\end{document}
