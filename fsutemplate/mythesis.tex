\documentclass[11pt]{fsuthesis}
% Imports provided with Evan's template
%\usepackage[overload]{textcase}
%\usepackage[dvips]{graphics}
\usepackage{hyperref}
%\hypersetup{breaklinks=true}

% Imports added by Ian
\usepackage{verbatim}
\usepackage{amsmath, amssymb, graphics}
\usepackage[pdftex]{graphicx}
\usepackage[numbers,sort&compress]{natbib}
\usepackage{textcomp} %supposedly needed by listings
\usepackage{subfig}






\input{macros.tex}
\input{misc_mac.tex}


\title{Real-Time Particle Systems in the Blender Game Engine}
\author{Ian Johnson}
\college{College of Arts and Sciences}
\department{Department of Scientific Computing}
\manuscripttype{Thesis}
\degree{Master of Science}
\semester{Summer}
\degreeyear{2011}
\defensedate{August 24, 2011}

\committeesig{Gordon Erlebacher}{Professor Directing Thesis}
\committeesig{Tomasz Plewa}{Committee Member}
\committeesig{Anter El-Azab}{Committee Member}

\collegesig{Gordon Erlebacher}{Chair}{Department of Scientific Computing}
\collegesig{Sam Huckaba}{Dean}{College of Arts and Sciences}

\begin{document}

\frontmatter
\maketitle
\makesignaturepage

\begin{dedication}
I dedicate this thesis to my parents David and Corine, who taught me both
explicitly and implicitly the value of education. All my life they have
fostered my curiousity and encouraged me to explore, for which I am eternally
grateful.
\end{dedication}

\begin{acknowledgments}
This thesis was made possible by the help and support of many people both near and far.
First and foremost I must thank my Advisor, Dr. Gordon Erlebacher for his
relentless support and encouragement. Dr. Erlebacher has an infectious love for
improvement and were it not for his help, both the code and my understanding
would be far short of where they are today. I thank him for training me to think like a scientist.

I must thank Dr. Anter El-Azab and Dr. Tomasz Plewa not only for taking the
time to be in my committee but for being excellent educators who have broadened
and deepend my understanding of scientific computing.

I would also like to thank my colleagues and friends in the Department of Scientific
Computing who have spent countless hours in the vizlab discussing, helping and
playing. Thank you Evan, Andrew, Nathan, Myrna, Steve and Olmo.

This project would not have been possible without the help and support of the
wonderful Blender community, thank you Moguri, dfelinto, Mike Pan and everyone
in the IRC channel, BlenderArtists and BlenderNation who helped and encouraged
me.

Finally I would like to thank my girlfriend Michelle for her support, including
but not limited to, making coffee for me all those late nights.
\end{acknowledgments}

\tableofcontents
%\listoftables
\listoffigures
%\listofmusex

%\begin{listofsymbols}
%\end{listofsymbols}

%\begin{listofabbrevs}
%\end{listofabbrevs}

\begin{abstract}
\begin{figure}[!htc]
 		\centering
		\includegraphics[scale=0.15]{figures/dsc_fsuseal.png}
		\label{fig:dsc_fsuseal}
\end{figure}

Advances in computational power have lead to many developments in science and
entertainment. Powerful simulations which required expensive supercomputers can
now be carried out on a consumer personal computer and many children and young
adults spend countless hours playing sophisticated computer games. The focus of
this research is the development of tools which can help bring the entertaining
and appealing traits of video games to scientific education.

One of the main factors contributing to the rise in computational power and the
increasing sophistication of video games is the Graphics Processing Unit (GPU).
As a piece of consumer hardware a GPU is relatively low cost and available to
almost anyone with a computer. The market for GPUs has been primarily driven by
video gamers wanting to improve the experience of their games through better
and often more realistic graphics. 

Video game developers use many tools and programming languages to build their
games, for example the Blender 3D content creation suite. Blender includes a
Game Engine that can be used to design and develop sophisticated interactive
experiences. One important tool in computer graphics and animation is the
particle system, which makes simulated effects such as fire, smoke and fluids
possible. The particle system available in Blender is unfortunately not
available in the Blender Game Engine because it is not fast enough to run in
real-time.

This thesis presents a particle system library integrated into the Blender Game
Engine providing an interactive platform for exploring fluid dynamics and
creating video games with realistic water effects.  The library is accelerated
by the GPU using the OpenCL programming language. The primary system
implemented in this research is a fluid simulator using the Smoothed Particle
Hydrodynamics technique for simulating incompressible fluids such as water.

The library created for this thesis can simulate water using SPH at
40fps with upwards of 100,000 particles on an NVIDIA GTX480 GPU. The
fluid system has interactive features such as object collision, and the
ability to add and remove particles dynamically. These features as well as
phsyical properties of the simulation can be controlled intuitively from the
user interface of Blender. 


\end{abstract}

\mainmatter

\input chapter_introduction
\input chapter_games
\input chapter_gpu
\input chapter_particles
\input chapter_sph
\input chapter_implementation
\input chapter_results
\input chapter_future

\pagebreak

\appendix
\input appendix_code
\input appendix_modified

%\toi{double check bib}

% You have your choice of bibliography sections, either
% hand-crafted or BibTeX.

% This is the "hand-crafted" bibliography/references section:
%\begin{references}
%Mybib, Sample.  \textit{An Example of a Bibligraphic Entry
% Created Manually}.  Tallahassee, Florida: Fornish and Frak, 2010.
%
%Smith, Marigold.  \textit{Lots and Lots of Bibliographic Entries
% and How to Display Them}.  Tallahassee, Florida: Gibson and Goulash, 2010.
%\end{references}

% The BibTeX bibliography/references section.  View the file
% 'myrefs.bib' to get a feel for what these entries may look like.
%\bibliographystyle{plain}
\bibliographystyle{unsrtnat}
\bibliography{mythesis}


\begin{biosketch}
    Ian 'enjalot' Johnson was born on the 6th of October, 1985 in Leiden,
    Netherlands. Ian grew up in Tallahassee, Florida where he became interested
    in computers at a young age. One chapter of his youth involved lucrative
    yet misguided efforts in hacking video games. Upon reaching some semblance
    of maturity Ian has channeled his creative energies into constructive
    projects and now enjoys building tools that help people create video games.

    Ian's research interests include computer graphics, scientific
    visualization, and interactive education. When not studying, he enjoys
    spending time with his loved ones, teaching those around him and traveling
    and eating local foods.

    Ian currently lives in Tallahassee, Florida but can be found from anywhere
    in the world by visiting his web-site \url{http://enja.org}.
\end{biosketch}

\end{document}
