\chapter{Particle Systems}

\section{Particle Systems}
A Particle system is a system composed of points in space whose behavior is
defined by rules that act on those points. Most commonly used in graphics and
video games to simulate special effects like fire, smoke, water and explosions
particle systems are also used in computational fluid dynamics, astrophysics
and materials science.

\begin{comment}
\begin{figure}[!htc]
 		%\centering
        % this picture should be wordwrapped I think
		\includegraphics[scale=0.5]{figures/pretty_particles.png}
		\label{fig:logic}
        \caption{ pretty particle system picture }
\end{figure}
\end{comment}
%\toi{put in a particle system picture}


All particle systems share some fundamental properties, first they must be able
to represent points in 1-, 2- or 3-dimensional space. For the purpose of this
research  3 dimensions will be used, with the location of each particle 
denoted with the coordinates \verb|x|, \verb|y| and \verb|z| in a Cartesian coordinate system. Each particle is
conceptualized as a unique individual in the system, usually representing some
small piece of a larger whole. In the fluid simulation of this research, each
particle represents some small volume of water, but for a fire effect the
particle serves as an animated paint brush. In both cases each particle will
have some associated properties unique to it, such as its density in the water
simulation, or its color in the fire effect.


A particle system is not useful unless the properties of it's particles change
in time, therefore it is important to have the concept of a \verb|time-step|.
The time-step determines how fast a system changes over time. It is important
to distinguish between simulation time and real-time. One can simulate the
formation of a galaxy over a million years, or the creation of a protein in
nanoseconds and render each as a 30 second movie. We will discuss simulation
time in terms of \verb|updates| to the system, where each update advances the
simulation time one time-step. The creator of the system can determine how many
updates to perform for some unit of real-time. This is usually measured in
frames per second, where a frame is the rendering of a 3D scene to be
displayed. A desirable frames per second (fps) for a video game is 60fps, with
30fps or above considered real-time. For reference, movies are often displayed
at 24fps. 


Each particle has a location and some properties associated with it, and these
can all change in time. Most particle systems follow physical rules, where the
location of each particle is updated according to forces acting upon it. In
these systems properties can include velocity, mass, temperature and other
physical concepts. The forces used to update the locations of the particles can
be external to the system, such as gravity or force fields (like wind) or
internal, meaning the particles exert force on each other based on their
properties. The forces are used to update the particle locations according to
Newton's laws of motion.


Particle systems are especially interesting because simple rules governing the
behavior of each particle can lead to complex behavior of the whole system.
Particle systems have been classified based on how the particles interact.
Non-interacting particle systems are simply systems wherein one particle can
not affect another, commonly used to simulate fireworks. Long range interacting
particle systems are systems where most of the particles interact with every
other particle, such as in gravitational $N$-body systems.  Short range
interacting particle systems are systems such as the fluid simulator
implemented in this research, where each particle interacts with neighbors in a
small region around it \cite{Knowles2009}.




\section{Introduction to the Framework}

There are many aspects to particle systems that are common to different
simulations, such as particle location and particle rendering. In order to
create an extensible tool, this project is designed as an object-oriented
software framework for general particle systems.  The primary motivation for
this is to avoid duplicating work when creating different types of simulations.


We first define an interface for the user of the particle system to interact
with, the \verb|RTPS| class. This class initializes a particle system with some
initial settings and exposes functions to update and render the particles. It
also allows access to it's internal system object, which has functions for
interacting with the system directly. These functions are general for all
systems, but may be implemented differently by different types of systems. For
example inserting particles to the system, or setting up a "hose" to 
spray particles.


The rendering of particles is handled separately from the simulation so that
different particle systems may take advantage of the available rendering
classes. It is also easy to add a new rendering algorithm by defining a new
rendering class derived from the \verb|Render| class.


Someone wishing to add a new type of particle system does not have to spend
much energy on infrastructure but can focus on implementing the rules
which govern the behavior of their particles, or can add new functionality to
interface with some external projects. However it is likely that users will want to
integrate particles into their existing game mechanics, and the framework has
been designed with this in mind.



