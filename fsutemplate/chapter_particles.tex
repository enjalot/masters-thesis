\chapter{Particle Systems}

\section{Particle Systems}
\toi{discuss the fundamental aspects of particle systems}
* representation as points in 1, 2 and 3 dimensional space
* per-particle properties
* per-particle forces
* particle - particle interaction


\section{Introduction to the Framework}

There are many aspects to particle systems which will be common to different
simulations, namely the locations of each particle in some space as well as the
notion of displaying them in some way. In order to create an extensible tool we
designed an object oriented software framework for general particle systems.
The primary motivation for this is to avoid duplicating work when creating
different simulations.

\section{The RTPS class}

The entry point into the framework is the RTPS.h header file. This file defines
the RTPS class which a user of the library will instanciate in order to run a
particle based simulation. The Application Programmers Interface (API) is
relatively simple, giving the user an object which only exposes two methods, a
\toi{come up with consistent way to highlight code inline}
$update$ function and an $render$ function. These two functions abstract
internal logic for updating the simulation and displaying of the particles. The
constructor to the class accepts either no arguments or an RTPSettings object.
This object contains many settings including the type of simulation to
instanciate, rendering options as well as simulation parameters. These settings
\toi{link to Fluid Simulation section}
will be covered in more detail in the Fluid Simulation section of the thesis.

The typical use case (with default settings) is envisioned as follows
\begin{cppcode}[0]
#include "RTPS.h"
using namespace rtps;
//... initialization phase ...
RTPSettings settings();
RTPS ps(settings);

//... run loop ...
while(true)
{
    //... event handling and game logic
    ps.update()
    //... rendering phase
    ps.render()
}

\end{cppcode}

\section{The System class}

Interaction with the simulation is provided through the $system$ object which
will have various methods depending on what type of simulation was set in the
settings. The most important methods are those which allow the user to add
particles to a system, namely $addBox$, $addBall$ and $addHose$. For the fluid
simulation there is also the $loadTriangles$ method which allows the user to
pass in triangles for the fluid particles to collide with.

These methods are declared in the System.h header file which defines an
Abstract Class $System$. The methods are implemented in the specific system
classes such as SPH and Simple.

\section{Common Structures}

Utilized throughout the code is the $float4$ struct. $float4$ structs are used
to represent coordinate positions of the particles, as well as many of the
properties associated with the particles such as density, velocity and force.
The functionality of the $float4$ class match the OpenCL $float4$ type as close
as possible. We have overloaded several operators for constructing and doing
arithmetic as well as provided a print function. This functionality allows the
developer to quickly construct data for interacting with the system. Being able
to reimplement an OpenCL routine on the CPU is also convenient for debugging
logic since it makes printing output much simpler. 
There is also the $int4$ struct which is similarly a vector of four integers
for more convenient communication with OpenCL.





