\chapter{Particle Systems}

\section{Particle Systems}
A Particle system is a system composed of points in space whose behavior is
defined by rules which act on those points. Most commonly used in graphics and
video games to simulate special effects like fire, smoke, water and explosions
particle systems are also used in computational fluid dynamics, astrophysics
and materials science.

\begin{comment}
\begin{figure}[!htc]
 		%\centering
        % this picture should be wordwrapped i think
		\includegraphics[scale=0.5]{figures/pretty_particles.png}
		\label{fig:logic}
        \caption{ pretty particle system picture }
\end{figure}
\end{comment}



All particle systems share some fundamental properties, first they must be able
to represent points in 1, 2 or 3 dimensional space. For the purpose of this
research  3 dimensions will be used, with the location of each particle being
denoted with the coordinates \verb|x|, \verb|y| and \verb|z|. Each particle is
conceptualized as an unique individual in the system, usually representing some
small piece of a larger whole. In the fluid simulation of this research each
particle represents some small volume of water, but for a fire effect the
particle serves as an animated paint brush. In both cases each particle will
have some associated properties unique to it, such as its density in the water
simulation, or its color in the fire effect.


A particle system is not useful unless it changes over time, therefore it is
important to have the concept of a \verb|time-step|. The time-step is used to
determine how fast a system changes over time. It is important to distinguish
between simulation time and real-time, one can simulate the formation of a
galaxy over a million years, or the creation of a protein in nanoseconds and
render each as a 30 second movie. We will discuss simulation time in terms of
\verb|updates| to the system, where each update advances the simulation time
one time-step. The creator of the system can determine how many updates to
perform for some unit of real-time. This is usually measured in frames per
second, where a frame is the rendering of a 3D scene to be displayed. A
desirable frames per second (fps) for a video game is 60fps, with 30fps
considered real-time. For reference, movies are displayed at 24fps. 


Each particle has a location and some properties associated with it, and these
can all change over time. The most common way to control how they change is by
calculating the forces exerted on each particle. These forces can be external
to the system, such as gravity or force fields (like wind) or internal, meaning
the particles exert force on each other. When we update the system, we want to
calculate a new location for each particle based on the forces applied to it.
This is done by integrating the force to obtain a velocity and then integrating
the velocity to obtain the new position.


Particle systems are especially interesting because simple rules governing the
behavior of each particle can lead to complex behavior of the whole system.
Particle systems have been classified into various classes based on how the
particles interact. Non-interacting particle systems are simply systems where
all of the forces are determined by the individual particle and no others are
considered, commonly used to simulate fireworks. Long range interacting
particle systems are systems where most of the particles interact with every
other particle, such as in gravitational N-body systems. Short range
interacting particle systems are systems such as the fluid simulator proposed
in this research, where each particle interacts with neighbors in a small
region around it. \cite{Knowles2009} 




\section{Introduction to the Framework}

There are many aspects to particle systems which will be common to different
simulations, namely the locations of each particle in some space as well as the
notion of displaying them in some way. In order to create an extensible tool 
this project designed an object oriented software framework for general particle systems.
The primary motivation for this is to avoid duplicating work when creating
different simulations.


We first define an interface for the user of the particle system to interact
with, the \verb|RTPS| class. This class initializes a particle system with some
initial settings and exposes functions to update and render the particles. It
also allows access to it's internal system object, which has functions for
interacting with the system directly. These functions are general for all
systems, but may be implemented differently by different types of systems. For
example inserting particles to the system, or setting up a "hose" which will
spray particles.


The rendering of particles is handled separately from the simulation so that
different particle systems may take advantage of the available rendering
classes. It is also easy to add a new way of rendering by defining a new
rendering class derived from the \verb|Render| class.


Someone wishing to add a new type of particle system does not have to spend
much energy on infrastructure but can rather focus on implementing the rules
which govern the behavior of their particles, or adding new functionality to
interface with some external projects. It is likely that users will want to
integrate particles into their existing game mechanics, and the framework has
been designed with this in mind.



