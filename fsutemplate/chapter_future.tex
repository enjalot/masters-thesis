\chapter{Future Work}

\section{Optimization}

The project demonstrates real-time performance on modern graphics hardware,
however there is always room for improvement. Several bottlenecks in the code
have known methods for elimination. Possibly the greatest speedup would involve
improving the neighbor search. A technique known as Z-indexing would allow for
neighbor lookups without storing the cell-indices array and consequently
cutting a significant amount of memory accesses \cite{Goswami2010}.


Another known bottleneck is the sorting algorithm used in the process of
preparing the neighbor search. Researchers have demonstrated a highly efficient
implementation of the Radix sort in CUDA \cite{Merrill:Sorting:2010} which can be ported to OpenCL.


Collision detection is currently implemented in a brute force manner. Data
structures must be investigated which can handle particle collisions with
objects at various scales. One such data structure is Hierarchical Spatial
Subdivision, an extension of the current grid-based method for particle
interaction \cite{Pouchol2009}. Triangle-based collisions are desirable for
their ability to handle general objects but are not necessary for all use
cases. Implementing various collision primitives such as spheres, cylinders and
axis aligned bounding boxes will allow for interesting interactions in game
play at a lower performance cost. 


\section{Rendering}
For the purpose of fluid effects in a game, it is important to consider how the
fluid is displayed. This is especially difficult in real-time, as traditional
rendering methods such as marching cubes have not yet been implemented at
suitable speeds.
Two approaches are being considered. First a screen-space technique that uses
curvature flow to give the impression of a fluid surface \cite{VanderLaan2009}.
Another approach is to extract the surface particles using data from the
simulation and performing GPU ray casting on the much smaller subset of
points \cite{Goswami2010}. Volume rendering techniques for the GPU have received
increasing interest and may present an opportunity for real-time
rendering \cite{Fraedrich2010}.


It is also desirable to support effects for other kinds of fluids such as smoke
and other particle system effects such as fire. For these techniques a variety
of blending modes and custom textures must be made available to the user.
Sorting particles with respect to depth from the camera is an essential task
for enabling proper alpha blending used by some of these effects.

\section{Blender}

The current integration with Blender gives a solid foundation for future
improvements. One important goal is to move the user interface from the
Modifiers to its own Particle System panel. This is desirable from the
perspective of current Blender users who are already familiar with the existing
particle system. Along these lines, another goal is to integrate with the
existing particle system as much as possible, both in user interface as well as
functionality.
Further improvements to the user interface will include a special modifier for
emitting particles from objects, eliminating the error prone and tedious method
of adding custom game properties. 


Another important goal is the creation of a Python interface to the RTPS
library within Blender and the Game Engine. Python scripting gives the game
designer powerful control over the behavior of their game and exposing RTPS
functions to Python will allow for new and creative uses. 


Currently collision detection and interaction with objects in the Game Engine
do not use all of the available functionality present. The code could be made
cleaner by utilizing existing classes for interacting with Objects as well as
improved communication with the Bullet physics engine.  

\section{SPH}

There are a host of improvements that can be made to the simulator itself,
including better physical accuracy, more physical phenomena and interesting
combinations of effects.
There are several features available in the literature that can be readily
implemented such as smoothing of field variables and improved smoothing kernels for better
stability \cite{Liu2010}, surface tension and visco-elastic
effects for more convincing and varied fluids \cite{Clavet2005}. Simulating
multiple fluids interacting at once as well as phase changes based on
temperature is also desirable \cite{Muller2005}. SPH can also be extended to
simulate granular materials such as sand \cite{Bell2005}.


A very important goal for the future is better interaction with solid
boundaries and rigid objects. Semi-analytic boundary conditions have been put
forth which allow for more physically accurate and computationally feasible
collisions.\cite{Manenti2011} \cite{Harada2007}
Interaction with rigid bodies has been explored by representing the solid
objects as a set of particles that are subject to fluid-like forces with rigid
constraints.\cite{Harada2007c} Fluids have also been coupled with cloth simulations with nice results.\cite{Harada2007b}



